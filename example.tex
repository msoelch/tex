% example document
% last updated 2017-02-23

\documentclass
[
%fontsize=11pt, % Document font size
twoside, % Shifts odd pages to the left for easier reading when printed, can be changed to oneside
%captions=tableheading,
%index=totoc,
%hyperref
]
{article}

\usepackage[orthodox]{nag}


% % % % % % % % % %
% MATH
% % % % % % % % % %
\usepackage[fleqn]{amsmath}	% fleqn for left alignment of math blocks (as in classicthesis)
\usepackage{amssymb}
\usepackage{amsfonts}
\usepackage{amsthm}
\newtheorem{definition}{Definition}%[chapter]
\newtheorem{theorem}{Theorem}%[chapter]
\newtheorem{corollary}{Corollary}%[chapter]


% % % % % % % % % %
% FLOATS, FIGURAES, AND TABLES
% % % % % % % % % %
\usepackage{graphicx} 		% Required for including images
\graphicspath{{gfx/}}
\usepackage{caption}
\captionsetup{font=small} 	% format=hang,
\usepackage{subfig}
\usepackage{tabularx}
\setlength{\extrarowheight}{3pt} % increase table row height
\usepackage{pbox} 			% for linebreaking in cells
\usepackage{multicol, multirow}
\usepackage{makecell}
\usepackage{tikz}
\usepackage{tikzsymbols}
\usetikzlibrary{fit,arrows,shapes}
\usetikzlibrary{decorations.pathreplacing}
\usetikzlibrary{shadows}
\tikzset{>=latex}
\tikzset{every picture/.style=thick}


% % % % % % % % % %
% MISC
% % % % % % % % % %
\usepackage
[
%disable
]{todonotes}
\usepackage[utf8]{inputenc} % allow utf-8 input
\usepackage[T1]{fontenc}    % use 8-bit T1 fonts
%\usepackage[english]{babel} % hyphenation, special characters, ...
\usepackage{csquotes}		% recommended with babel for quotes
\usepackage[hidelinks,breaklinks]{hyperref}       % hyperlinks
\usepackage{cleveref}		% provides \cref for nice in-document refs
\usepackage{url}            % simple URL typesetting
\usepackage{nicefrac}       % compact symbols for 1/2, etc.
\usepackage{booktabs}
%\usepackage
%[
%activate={true,nocompatibility}, % activate protrusion and expansion
%final,						% enable microtype; use "draft" to disable
%tracking=true,
%kerning=true,
%spacing=true,
%factor=1100,				% add 10% to the protrusion amount (default: 1000)
%stretch=10,					% reduce stretchability (default: 20)
%shrink=10					% reduce shrinkability (default: 20)
%]
%{microtype}
%\microtypecontext{spacing=nonfrench}
\usepackage{cancel}			% easy cancelling of terms: \cancel{expression}
\usepackage[shortcuts]{extdash} % use, e.g., \-/ for hyphen within a word to indicate that the word may be broken here
\usepackage[printonlyused]{acronym}
\renewcommand{\acffont}[1]{\emph{#1}}
\renewcommand{\acfsfont}[1]{\normalfont #1}
\newcommand{\accite}[2]{\acffont{\acl{#1}} (\acsu{#1}; \cite{#2})\xspace}
\newcommand{\acpcite}[2]{\acffont{\aclp{#1}} (\acsp{#1}\acused{#1}; \cite{#2})\xspace}

\usepackage{mlmacros}		% custom commmands, mostly math

\variables{x,y,A}
\variables[mean,std]{\mu,\sigma}
\varmacros{E}{\mathcal{E}}

\probdists{p,q}
\probdists[policy]{\pi}
\MkProbDist{P}{\mathbb{P}}


\title{Documentation for \texttt{mlmacros.sty}}
\author{Maximilian Soelch}

\begin{document}
\maketitle
The \texttt{mlmacros} package provides convenient and readable macros for commonly used syntax when writing scientific articles about machine learning.

\section{\texttt{\textbackslash variables}}
Macros can help a great deal to stick with keeping notational conventions while maintaining readability of the source file. However, with an increasing number of conventions, maintenance of the macros becomes just as hard.

The \texttt{\textbackslash variables} macro eases some of that pain. It is called in the preamble with a list of variables and dynamically creates macros based on this list. For example, the call \texttt{\textbackslash variables\{x,y,A\}} automatically provides the following macros:

\begin{table}[hb]
	\centering
	\begin{tabular}{cccccccccc}
		\texttt{x} & \texttt{\textbackslash bx}& \texttt{\textbackslash xt} & \texttt{\textbackslash bxt} &\texttt{\textbackslash xT} & \texttt{\textbackslash bxT} & \texttt{\textbackslash xts} & \texttt{\textbackslash bxts} & \texttt{\textbackslash xTs} & \texttt{\textbackslash bxTs}\\
					& \bx & \xt & \bxt & \xT & \bxT & \xts  & \bxts & \xTs & \bxTs\\
		\texttt{y} & \texttt{\textbackslash by}& \texttt{\textbackslash yt} & \texttt{\textbackslash byt} &\texttt{\textbackslash yT} & \texttt{\textbackslash byT} & \texttt{\textbackslash yts} & \texttt{\textbackslash byts} & \texttt{\textbackslash yTs} & \texttt{\textbackslash byTs}\\
		& \by & \yt & \byt & \yT & \byT & \yts  & \byts & \yTs & \byTs\\
		\texttt{A} & \texttt{\textbackslash bA}& \texttt{\textbackslash At} & \texttt{\textbackslash bAt} &\texttt{\textbackslash AT} & \texttt{\textbackslash bAT} & \texttt{\textbackslash Ats} & \texttt{\textbackslash bAts} & \texttt{\textbackslash ATs} & \texttt{\textbackslash bATs}\\
		& \bA & \At & \bAt & \AT & \bAT & \Ats  & \bAts & \ATs & \bATs
	\end{tabular}
\end{table}

The driving idea behind these macros is the application to time series of scalars (normal font), vectors (bold font) and matrices (upper-case bold font). These macros can easily be edited and/or extended by adjusting the \texttt{mlmacros.sty} file.

It even works with more complicated replacements, such as greek letters or other math symbols. Calling \texttt{\textbackslash variables[mean, std]\{\textbackslash mu,\textbackslash sigma\}} provides analogous commands such as \texttt{\textbackslash bmeanTs} for $\bmeanTs$ or \texttt{\textbackslash stdts} for $\stdts$.

More complicated macro substitutions currently do not work with \texttt{\textbackslash variables}. However, you can exploit the underlying command \texttt{\textbackslash varmacros} for every pair: \texttt{\textbackslash varmacros\{E\}\{\textbackslash mathcal\{E\}\}} for commands like \texttt{\textbackslash Et}, which yields $\bEt$.

\section{\texttt{\textbackslash probdists}}\label{sec:probdists}
The \texttt{\textbackslash probdists} macro provides a rich macro for probability distributions.

The call \texttt{\textbackslash probdists\{p\}} provides a command that can be used in any of the following ways:

\begin{table}[hb]
	\centering
	\begin{tabular}{ccccccccc}
		\texttt{\textbackslash p}& \texttt{\textbackslash p\{x\}} & \texttt{\textbackslash p\{x\}\{z\}} &\texttt{\textbackslash p[y]\{x\}\{z\}} &\texttt{\textbackslash p[y]\{x\}} & \texttt{\textbackslash p[y]}\\
		\p & \p{x} & \p{x}{z} & \p[y]{x}{z} & \p[y]{x} & \p[y]
	\end{tabular}
\end{table}
The \texttt{\textbackslash p} macro dynamically decides which parts to display. This allows for rapid adjustment of the syntax. The parentheses adjust dynamically via internal usage of \texttt{\textbackslash mleft} and \texttt{\textbackslash mright},
\eq{\p{\frac{x}{y}},}
which can be undone by starring the command, \texttt{\textbackslash p*\{\textbackslash frac xy\}} leads to
\eq{ \leadsto \p*{\frac xy}.}

Moreover, the \texttt{\textbackslash p} macro ignores any argument not given in set parentheses, \eg \texttt{\textbackslash p\{x\}z} yields $\p{x}z$.

Of course, \texttt{probdists} can be called with a list of letters, \eg \texttt{\textbackslash probdists\{p,q\}}. Similarly as the \texttt{\textbackslash variables} macro, we can use, \eg Greek letters, by passing optional arguments: \texttt{\textbackslash probdists[policy]\{\textbackslash pi\}} provides macros such as \texttt{\textbackslash policy\{u\}\{x\}} for $\policy{u}{x}$.

At this point, more complicated constructions like $\mathbb{P}$ can be achieved by calling the underlying \texttt{\textbackslash MkProbDist}, \eg \texttt{\textbackslash MkProbDist\{P\}\{\textbackslash mathbb\{P\}\}} for $\P{x}$.

\emph{Warning:} The macros provided by \texttt{probdists} overwrite any previously created macro of the same name.

\section{\texttt{\textbackslash expc} and \texttt{\textbackslash var}}
For expectation and variance, this package provides overloaded macros similar to the dynamic macros provided by \texttt{\textbackslash probdists} in \cref{sec:probdists}.

The application is straightforward:
\begin{table}[hb]
	\centering
	\begin{tabular}{cccc}
		\texttt{\textbackslash expc\{x\}} & \texttt{\textbackslash expc\{x\}\{z\}} &\texttt{\textbackslash expc[y]\{x\}\{z\}} &\texttt{\textbackslash expc[y]\{x\}}\\
		 \expc{x} & \expc{x}{z} & \expc[y]{x}{z} & \expc[y]{x} \\
		\texttt{\textbackslash var\{x\}} & \texttt{\textbackslash var\{x\}\{z\}} &\texttt{\textbackslash var[y]\{x\}\{z\}} &\texttt{\textbackslash var[y]\{x\}} \\
		\var{x} & \var{x}{z} & \var[y]{x}{z} & \var[y]{x}
	\end{tabular}
\end{table}

All comments from \cref{sec:probdists} apply here as well.

\section{Drafting a paper}
This section collects some macros on top of the \texttt{todonotes} package.
When a reference is missing in the text, you can use \texttt{\textbackslash addref}\addref.
You can also add a list argument that will be turned into bullets\addref{missingref1,missingref2}.

Following \href{http://lesswrong.com/lw/ix/say_not_complexity/}{$\rightarrow$this blog post by Eliezer Yudkowsky}, it's a good practice to say ``And then \magically, we get A'', or ``by \magic, B follows'' to make the missing links and arguments explicit.
This package provides the two corresponding macros \texttt{\textbackslash magic}\texttt{\textbackslash magically}, which can also be used like \texttt{\textbackslash magic \{idea to fix this\}} to yield \magic{idea to fix this}.


\section{Syntactic sugar}
\subsection{Math}
Some general, commonly used math expressions are provided by short, readable macros:
\begin{itemize}
	\item \texttt{mathbb} wrappers: \texttt{\textbackslash NN} for \NN, \texttt{\textbackslash RR} for \RR, \texttt{\textbackslash PP} for \PP, \texttt{\textbackslash EE} for \EE.
	\item \texttt{mathcal} wrappers: \texttt{\textbackslash mcX} for \mcX, \texttt{\textbackslash mcU} for \mcU, \texttt{\textbackslash mcZ} for \mcZ, \texttt{\textbackslash mcD} for \mcD.
	\item \texttt{\textbackslash data} for \data.
	\item \texttt{\textbackslash ind} for \ind.	
	\item \texttt{\textbackslash left\textbackslash right} wrappers: \eq{\texttt{\textbackslash abs\{\}}:\abs{\frac a b} \qquad\texttt{\textbackslash set\{\}}:\set{\frac ab}	\qquad\texttt{\textbackslash interval\{\}}:\interval{\frac ab}}
	\item Loss functions: \texttt{\textbackslash loss[index]\{arg\}} for $\loss[index]{arg}$. The parentheses adjust to the height of the arguments.
	\item Kullback-Leibler divergence: \texttt{\textbackslash kl\{p\}\{q\}} for $\kl{p}{q}$. The parentheses adjust to the height of the arguments.
	\item Gaussian/Normal distributions: \texttt{\textbackslash gauss\{0,1\}} for $\gauss{0, 1}$. The parentheses adjust to the height of the arguments unless \texttt{\textbackslash gauss*} is used. Overloaded, can be used for conditionals as well: \texttt{\textbackslash gauss\{x\}\{\textbackslash mu, \textbackslash sigma\}} gives $\gauss{x}{\mu, \sigma}$.
	\item Integrals: \texttt{\textbackslash dint} for nicer integrals: \eq{\texttt{\textbackslash int x \textbackslash dint x}: \;\int x \dint x \qquad\quad \texttt{\textbackslash int x dx}:\;\int x dx}
	\item Equations that are to be proven: \texttt{x \textbackslash shallbe y} for $x \shallbe y$. 
\end{itemize}
\subsection{Other}
One of the most cumbersome typesetting issues is the correct spacing after abbreviations.

This packages provides macros that automatically apply correct spacing and follow \href{http://www.quickanddirtytips.com/education/grammar/ie-versus-eg?page=2}{$\rightarrow$this guide} for 
\begin{itemize}
	\item \emph{with respect to}: \texttt{derivative \textbackslash wrt input} $\leadsto$ derivative \wrt input,
	\item \emph{exempli gratia}: \texttt{algrorithms, \textbackslash eg EM} $\leadsto$ algorithms, \eg EM; also \texttt{\textbackslash Eg} for \Eg
	\item \emph{id est}: \texttt{156 members, \textbackslash ie almost all} $\leadsto$ 156 members, \ie almost all; also \texttt{\textbackslash Ie} for \Ie
	\item \emph{et cetera}: \texttt{Germany, France, \textbackslash etc} $\leadsto$ Germany, France, \etc
\end{itemize}
The latter command checks if a period follows, \ie \texttt{\textbackslash dots Germany, France, \textbackslash etc. Other countries \textbackslash dots} yields \dots Germany, France, \etc Other countries \dots

\section{Style of this document}
The default style of this document uses Palantino as the main font via \eq{\texttt{\textbackslash usepackage[osf,sc]\{mathpazo\}{}}\\\texttt{\textbackslash linespread\{{1.05}\}\textbackslash selectfont}} with some extra spacing between lines. Moreover, the Euler math font is used: \eq{\texttt{\textbackslash usepackage[euler-digits]\{{eulervm}\}}} The \texttt{amsmath} package is loaded with the \texttt{fleqn} for left flush of equations. 

All these changes were taken from the highly recommended \href{https://bitbucket.org/amiede/classicthesis/wiki/Home}{$\rightarrow$\texttt{classicthesis}} package by Andr\'e Miede.

Other recommended nice-to-have packages for convenient and good-looking typewriting are \texttt{cleveref} for extremely clever referencing, \texttt{nicefrac} for better handling of inline fractions, and \texttt{microtype}, which takes care of small typesetting issues that improve the overall document appearance.

\section{\TeX studio autocomplete}
Along with this style file and example document, a \texttt{cwl} file is provided. By adding it to your \TeX studio, autocomplete for the static macros is provided. At this point, autocomplete of the dynamically produced \texttt{variables} and \texttt{probdists} is not included.

Under Windows 10, it has to be copied to the directory \eq{\texttt{\%appdata\%\textbackslash texstudio\textbackslash completion\textbackslash user}.} 

Under Ubuntu 16.04, it has to be copied to the directory \eq{\texttt{\textasciitilde/.config/texstudio}.} 



\end{document}