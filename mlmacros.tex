% example document
\PassOptionsToPackage{table}{xcolor} % xcolor tends to produce option clashes because it is imported from all sorts of packages. This forces the table option to be inserted everywhere. https://tex.stackexchange.com/a/5375
\documentclass
[
%fontsize=11pt, % Document font size
%twoside, % Shifts odd pages to the left for easier reading when printed, can be changed to oneside
%captions=tableheading,
%index=totoc,
%hyperref
]
{article}
\usepackage[orthodox]{nag}


% % % % % % % % % %
% MATH
% % % % % % % % % %
\usepackage[fleqn]{amsmath}	% fleqn for left alignment of math blocks (as in classicthesis)
\usepackage{amssymb}
\usepackage{amsfonts}
\usepackage{amsthm}
\newtheorem{definition}{Definition}%[chapter]
\newtheorem{theorem}{Theorem}%[chapter]
\newtheorem{corollary}{Corollary}%[chapter]


% % % % % % % % % %
% FLOATS, FIGURAES, AND TABLES
% % % % % % % % % %
\usepackage{graphicx} 		% Required for including images
\graphicspath{{gfx/}}
\usepackage{caption}
\captionsetup{font=small} 	% format=hang,
\usepackage{subfig}
\usepackage{tabularx}
\setlength{\extrarowheight}{3pt} % increase table row height
\usepackage{pbox} 			% for linebreaking in cells
\usepackage{multicol, multirow}
\usepackage{makecell}
\usepackage{tikz}
\usepackage{tikzsymbols}
\usetikzlibrary{fit,arrows,shapes}
\usetikzlibrary{decorations.pathreplacing}
\usetikzlibrary{shadows}
\tikzset{>=latex}
\tikzset{every picture/.style=thick}


% % % % % % % % % %
% MISC
% % % % % % % % % %
\usepackage
[
%disable
]{todonotes}
\usepackage[utf8]{inputenc} % allow utf-8 input
\usepackage[T1]{fontenc}    % use 8-bit T1 fonts
%\usepackage[english]{babel} % hyphenation, special characters, ...
\usepackage{csquotes}		% recommended with babel for quotes
\usepackage[hidelinks,breaklinks]{hyperref}       % hyperlinks
\usepackage{cleveref}		% provides \cref for nice in-document refs
\usepackage{url}            % simple URL typesetting
\usepackage{nicefrac}       % compact symbols for 1/2, etc.
\usepackage{booktabs}
%\usepackage
%[
%activate={true,nocompatibility}, % activate protrusion and expansion
%final,						% enable microtype; use "draft" to disable
%tracking=true,
%kerning=true,
%spacing=true,
%factor=1100,				% add 10% to the protrusion amount (default: 1000)
%stretch=10,					% reduce stretchability (default: 20)
%shrink=10					% reduce shrinkability (default: 20)
%]
%{microtype}
%\microtypecontext{spacing=nonfrench}
\usepackage{cancel}			% easy cancelling of terms: \cancel{expression}
\usepackage[shortcuts]{extdash} % use, e.g., \-/ for hyphen within a word to indicate that the word may be broken here
\usepackage[printonlyused]{acronym}
\renewcommand{\acffont}[1]{\emph{#1}}
\renewcommand{\acfsfont}[1]{\normalfont #1}
\newcommand{\accite}[2]{\acffont{\acl{#1}} (\acsu{#1}; \cite{#2})\xspace}
\newcommand{\acpcite}[2]{\acffont{\aclp{#1}} (\acsp{#1}\acused{#1}; \cite{#2})\xspace}
\usepackage{mlmacros}		% custom commmands, mostly math
% These packages are very nice features for a note.
% However, they are typically incompatible with a conference style.
% Hence, they are separated into a second header file

% % % % % % % % % %
% FONTS & STYLE
% % % % % % % % % %
\usepackage[osf,sc]{mathpazo} % Palatino as the main font
\linespread{1.05}\selectfont % Palatino needs some extra spacing, here 5% extra
%%\usepackage[scaled=0.85]{beramono}
\usepackage[euler-digits]{eulervm} % nicer math font
\renewcommand{\mathbf}{\mathbold} % necessary for greek letters in this math font to be displayed correctly
\usepackage[parfill]{parskip} % blank lines between paragraphs instead of indent

\usepackage[
%showframe,
paperwidth=210mm,
paperheight=297mm,
left=80pt,
top=80pt,
textwidth=350pt,
marginparsep=20pt,
marginparwidth=124pt,
textheight=692pt,
footskip=50pt
]
{geometry}

% % % % % % % % % %
% BIBLIOGRAPY
% % % % % % % % % %


\usepackage[
	backend=biber, %instead of bibtex
	%backend=bibtex8,
	bibencoding=ascii,%
	language=auto,%
	style=authoryear,%
	%style=authoryear-comp, % Author 1999, 2010
	%bibstyle=authoryear,dashed=false, % dashed: substitute rep. author with ---
	sorting=nyt, % name, year, title
	maxbibnames=4, % default: 3, et al.
	%backref=true,%
	natbib=true % natbib compatibility mode (\citep and \citet still work)
]{biblatex}
% alternative: \usepackage{natbib}

\addbibresource{example.bib}

% % % % % % % % % % % % % % % % % % % %
% new custom commands go here
\variables{x,y,A}
\variables[mean,std]{\mu,\sigma}
\varmacros{E}{\mathcal{E}}

\probdists{p,q}
\probdists[policy]{\pi}
\MkProbDist{P}{\mathbb{P}}

\usepackage{fancyvrb}
% % % % % % % % % % % % % % % % % % % %

\title{Documentation for \texttt{mlmacros.sty}}
\author{Maximilian Soelch}

\begin{document}
\maketitle
The \verb|mlmacros| package provides convenient and readable macros for commonly used syntax when writing scientific articles about machine learning.

\section{Subindexing sequence data}
When writing about algorithms on sequence data, we often refer to variables at a specific time by subindexing, \eg $x\tsub{t}, x\tm,x\Tp{2}$, or $x\tsub{i}{j} = (x_i, \dots, x_j)$ for a sequence.
This package provides shortcuts for the most common indexes:

\begin{table}[h!]
	\centering
	\begin{tabular}[t]{lllll}
		\verb|x\tm| & \verb|x\tp| & \verb|x\ts| & \verb|x\tsm| & \verb|x\tsp|\\
		$x\tm$ & $x\tp$ & $x\ts$ & $x\tsm$ & $x\tsp$\\[1.5em]
		
		\verb|x\tm{k}| & \verb|x\tp{k}| & & \verb|x\tsm{k}| & \verb|x\tsp{k}|\\
		$x\tm{k}$ & $x\tp{k}$ & & $x\tsm{k}$ & $x\tsp{k}$\\[1.5em]

		\verb|x\Tm| & \verb|x\Tp| & \verb|x\Ts| & \verb|x\Tsm| & \verb|x\Tsp|\\
		$x\Tm$ & $x\Tp$ & $x\Ts$ & $x\Tsm$ & $x\Tsp$\\[1.5em]

		\verb|x\Tm{k}| & \verb|x\Tp{k}| & & \verb|x\Tsm{k}| & \verb|x\Tsp{k}|\\
		$x\Tm{k}$ & $x\Tp{k}$ & &  $x\Tsm{k}$ & $x\Tsp{k}$\\[1.5em]
	
		\verb|x\tsub{k}| & & & \verb|x\tsub{i}{j}| & \\		
		$x\tsub{k}$ & & & $x\tsub{i}{j}$ & 
	\end{tabular}
\end{table}
The logic behind the commands is the following:
\begin{itemize}
	\item Start with the desired upper or lower case \verb|t|.
	\item Optionally add an \verb|s| for a sequence starting at $1$.
	\item Add \verb|m| or \verb|p| for minus or plus.
	\item Optionally add an argument to replace the default $1$.
\end{itemize}

Notice that the plus and minus signs are slightly shortened, cf.
\begin{align*}
	&x_{t-1:T+K}\\
	&x\tsub{t\shortminus1}{T\shortplus K}
\end{align*}
Use \verb|\shortminus| or \verb|\shortplus| to make use of these.
Alternatively, you can  use \verb|\tminus| or \verb|\tplus| instead of \verb|t\shortminus| or \verb|t\shortplus|, respectively.

\section{\texttt{\textbackslash variables}}
Macros can help a great deal to stick with conventions while maintaining readability of the source file. 
However, with an increasing number of conventions, maintenance of the macros becomes just as hard.

The \verb|\variables| macro eases some of that pain. 
Call it in the preamble with a list of variables and dynamically creates macros based on this list. 
For example, the call \verb|\variables{x}| automatically provides the following macros:

\begin{table}[h!]
	\centering
	\begin{tabular}{llllll}

		\verb|\x| & \verb|\xseq| & \verb|\xall| & \verb|\xpast| & \verb|\xfilter| & \verb|\xfuture|\\
		\x & \xseq & \xall & \xpast & \xfilter & \xfuture\\[1.5em]		

		\verb|\bx| & \verb|\bxseq| & \verb|\bxall| & \verb|\bxpast| & \verb|\bxfilter| & \verb|\bxfuture|\\
		\bx & \bxseq & \bxall & \bxpast & \bxfilter & \bxfuture
	\end{tabular}
\end{table}

It works with more complicated replacements, such as greek letters or other math symbols. 
Calling \verb|\variables[mean,std]{\mu,\sigma}| provides commands such as \verb|\mean| for $\mean$, \verb|\bmeanseq| for $\bmeanseq$ or \verb|\stdall| for $\stdall$.
Notice that the list delimiter currently needs to be a comma \emph{without enclosing spaces}.

More complicated macro substitutions currently do not work with \verb|\variables|. However, you can exploit the underlying command \verb|\varmacros| for every pair: \verb|\varmacros{E}{\mathcal{E}}| for commands like \verb|\bE|, which yields $\bE$.

Further automatically created macros can easily be edited and/or extended by adjusting the \verb|mlmacros.sty| file.

\section{\texttt{\textbackslash probdists}}\label{sec:probdists}
The \verb|\probdists| macro provides a rich macro for probability distributions.
The call \verb|\probdists{p}| provides a command that can be used in any of the following ways:

\begin{table}[h!]
	\centering
	\begin{tabular}{llllll}
		\verb|\p| & \verb|\p{x}| & \verb=\p{x|z}= & \verb=\p[y]{x|z}= & \verb|\p[y]{x}| & \verb|\p[y]|\\
		\p & \p{x} & \p{x|z} & \p[y]{x|z} & \p[y]{x} & \p[y]
	\end{tabular}
\end{table}
The \verb|\p| macro dynamically decides which parts to display. This allows for rapid adjustment of the syntax. The parentheses adjust dynamically via internal usage of \verb|\mleft| and \verb|\mright|, unless starred commands (\verb|\p*| \etc) are used,
\begin{equation}
	\p{x^2 | y_t} \text{ vs. starred } \p*{x^2 | y_t}.
\end{equation}

\verb|\p| ignores arguments not given in braces, \eg \verb|\p{x}z| yields $\p{x}z$.

Of course, \verb|probdists| can be called with a list of letters, \eg \verb|\probdists{p,q}|. 
Similarly as the \verb|\variables| macro, we can use, \eg Greek letters, by passing optional arguments: 
\verb|\probdists[policy]{\pi}| provides macros such as \verb=\policy{u|x}= for $\policy{u|x}$.

At this point, complicated constructions like $\mathbb{P}$ can be achieved by calling the underlying \verb|\MkProbDist|, \eg \verb|\MkProbDist{P}{\mathbb{P}}| for $\P{x}$.

\emph{Warning:} The macros provided by \verb|probdists| overwrite any previously created macro of the same name.

\section{\texttt{\textbackslash expc}, \texttt{\textbackslash var}, \texttt{\textbackslash loss}, \texttt{\textbackslash cost}, \texttt{\textbackslash entropy}}
For expectation and variance, this package provides overloaded macros similar to the dynamic macros provided by \texttt{\textbackslash probdists} in \cref{sec:probdists}.

The application is straightforward:

\begin{table}[h!]
	\centering
	\begin{tabular}{llll}
		\verb|\expc{x}| & \verb+\expc{x|z}+ &\verb+\expc[y]{x|z}+ &\verb|\expc[y]{x}|\\
		 \expc{x} & \expc{x|z} & \expc[y]{x|z} & \expc[y]{x} \\
	\end{tabular}
\end{table}
and analogously

\begin{table}[h!]
	\centering

	\begin{tabular}{*{3}{l}}

		\verb+\var[y]{x|z}+ & \verb+\loss[y]{x|z}+ & \verb+\cost[y]{x|z}+\\
		\var[y]{x|z} & \loss[y]{x|z} & \cost[y]{x|z} \\[1em]
		\verb+\entropy[y]{x|z}+&  \verb+\mi[y]{x|z}+ \\
		\entropy[y]{x|z} & \mi[y]{x|z}
	\end{tabular}
\end{table}
All comments from \cref{sec:probdists} apply here as well.

\section{Syntactic sugar}
\subsection{Math}
Some general, commonly used math expressions are provided by short, readable macros:
\begin{itemize}
	\item \verb|mathbb| wrappers: \verb|\NN| for \NN, \verb|\RR| for \RR, \verb|\PP| for \PP, \verb|\EE| for \EE.
	\item \verb|mathcal| wrappers: \verb|\mcX| for \mcX, \verb|\mcU| for \mcU, \verb|\mcZ| for \mcZ, \verb|\mcD| for \mcD.
	\item \verb|\data| for \data.
	\item \verb|\ind| for \ind.	
	\item \verb|\left\right| wrappers:
		\begin{itemize}
			\item \verb|\abs{}|: $ \abs{\frac a b }$ 
			\item \verb|\set{}|: $ \set{\frac ab }$	
			\item\verb|\interval{}|: $ \interval{\frac ab} $
		\end{itemize}
	\item Divergences: \verb|\divergence[a]{b}{c}| for $\divergence[a]{b}{c}$
	\item Kullback-Leibler divergence: \verb|\kl{p}{q}| for $\kl{p}{q}$. 
	The parentheses adjust to the height of the arguments unless you use \verb|\kl*|.
	\item Gaussian/Normal distributions: \verb|\gauss{0,1}| for $\gauss{0, 1}$. The parentheses adjust to the height of the arguments unless \verb|\gauss*| is used. 
	Can be used for conditionals as well: \verb+\gauss{x | \mu, \sigma}+ gives $\gauss{x | \mu, \sigma}$.
	\item Integrals: \verb|\dint| for nicer integrals.
	
		\verb|\int x \dint x| for $\int x \dint x$ instead of \verb|\int x dx|:$\int x dx$
	\item Equations that are to be proven: \verb|x \shallbe y| for $x \shallbe y$. 
	\item Transpositions: \verb|A\transpose| for $A\transpose$ (with minor spacing adjustments compared to $A^\top$).
	\item Cancelling: use \verb|\cancelto{a}{b}| for $\cancelto{a}{b}$
	\item Superscripts: use \verb|x\super{i}| for $x\super{i}$
	\item Gradients: use \verb|\grad\_{\theta}| for $\grad_{\theta}$ for better spacing than $\nabla_\theta$
\end{itemize}
\subsection{Abbreviations}
One of the most cumbersome typesetting issues is the correct spacing after abbreviations.
The following abbreviations are provided by \verb|mlmacros|:
\begin{center}
	\begin{tabular}{lll}
		\textsc{Abbrev.} & \textsc{Macro} & \textsc{Meaning}\\[.5em]
		\cf & \verb|\cf| & conferatur; compare\\
		\dof & \verb|\dof| & degrees of freedom\\
		\eg, \Eg & \verb|\eg|, \verb|\Eg| & exempli gratia; for instance\\
		\etal & \verb|\etal| & et alii; and others\\
		\etc & \verb|\etc| & et cetera\\
		\ie, \Ie & \verb|\ie|, \verb|\Ie| & id est; that is\\
		\iid & \verb|\iid| & independent and identically distributed\\
		\NB & \verb|\NB| & Nota Bene; note well\\
		\vs & \verb|\vs| & versus\\
		\wrt & \verb|\wrt| & with respect to
	\end{tabular}
\end{center}
All these macros check for a following dot.

To understand the difference, compare the spacing in

\begin{tabular}{l}
	\verb|There are examples, e.\ g.\ this, that and the other.|\\
	There are examples, e. g. this, that and the other.\\
	\verb|There are examples, \eg this, that and the other.|\\
	There are examples, \eg this, that and the other.
\end{tabular}

\subsection{Other}
For easy use in, \eg, tables: \verb|\xmark| and \verb|\cmark| for \xmark and \cmark.

\section{Style of this document}
The default style of this document uses Palantino as the main font via

\begin{Verbatim}[tabsize=4]
	\usepackage[osf,sc]{mathpazo}{}
	\linespread{{1.05}}\selectfont
\end{Verbatim}

with some extra spacing between lines. Moreover, the Euler math font is used:

\begin{Verbatim}[tabsize=4]
	\verb|\usepackage[euler-digits]{{eulervm}}| 
\end{Verbatim}

The \verb|amsmath| package is loaded with the \verb|fleqn| for left flush of equations. 

All these changes were taken from the highly recommended \href{https://bitbucket.org/amiede/classicthesis/wiki/Home}{$\rightarrow$\texttt{classicthesis}} package by Andr\'e Miede.

Other recommended nice-to-have packages for convenient and good-looking typewriting are \verb|cleveref| for extremely clever referencing, \verb|nicefrac| for better handling of inline fractions, and \verb|microtype|, which takes care of small typesetting issues that improve the overall document appearance.

\section{\TeX studio autocomplete}
Along with this style file and example document, a \verb|cwl| file is provided. 
By adding it to your \TeX studio, autocomplete for the static macros is provided. 
Autocompletion of the dynamically produced \verb|variables| and \verb|probdists| is not included.

Under Windows 10, it has to be copied to the directory 

\begin{Verbatim}
	%appdata%\texstudio\completion\user.
\end{Verbatim}

Under Ubuntu 16.04 and 18.04, it has to be copied to the directory 

\begin{Verbatim}
	~/.config/texstudio}.
\end{Verbatim} 

\end{document}