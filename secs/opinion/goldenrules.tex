\section{Golden Rules}

\paragraph{There is no such thing as too many macros.} 
Macros can feel like overkill at first.
After all, you are only writing a note, are you not?
Then the note turns into a paper.
There will always be a point where you change your mind with respect to notation.
And if you do it right, your source code becomes much more readable, particularly math equations.

\paragraph{Space is everything.}
A lot of what \LaTeX\ does is spacing your content gracefully.
Gradually grasping the inner workings is key to writing nice documents but also understanding when it fails.

A simple example is the difference between the dot after an acronym as in \etc and the full stop at the end of a sentence.
The space between sentences is larger than that between words.
You need to tell \LaTeX\ the dot after \etc is not a full stop.
Otherwise it looks awkwared, e. g., if you write like this instead of, \eg, like this.
See the difference?

\paragraph{Avoid hard-coding.}

Avoid hard-coding a certain look of the document, be it via setting specific lengths, positions, font modifications.
Instead, use macros, packages, commands, variables.
Leave as much flexibility to the compiler as possible and your document will have a higher chance of looking good.

There will be advanced examples below, here is one for illustration: you want to \emph{emphasize} a word?
Instead of italics, \verb|\textit|, use \verb|\emph|.
By default, it will use italics---but you can change your mind! And, while we're at it, \verb|\emph| \textit{changes \emph{behavior} in italics!}

\paragraph{Stackoverflow is your friend.}
\LaTeX\ has a steep learning curve.
The key is to learn how to search the right queries.
There are very few new problems in \LaTeX.
You'll get the hang of it!

\paragraph{Be consistent.}
Many of the suggestions in this document are only that---suggestions.
Whether you choose to adhere to them is often a matter of personal taste.
If you choose to follow some different standard, the key is to be conscious and consistent.