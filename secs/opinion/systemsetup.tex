\section{System Setup}

\paragraph{Distribution} 
Unless you are on a tight disk space budget, install a \TeX\ distribution with lots of included packages like \verb|texlive-full| for Linux.

\paragraph{Compiler} 
The best default for compiling your document is the \verb|latexmk -pdf| command, which should be included in your \TeX\ distribution.
It can detect which parts of the infamous multiple build runs, including the bibliography, need to be run based on changes in the files.

\paragraph{Editor}
Using a versatile plugin or dedicated \TeX\ editor is recommended.
I personally use \TeX Studio.
It is good-not-great, so you may find a solution that suits your preferences better.
I have found its following features useful:
\begin{itemize}
	\item Side-by-side display of source and pdf, which are tightly coupled so one can jump from the pdf to the respective source and vice versa.
	\item Easy setup of compiler and bibliography tool, including support of the aforementioned \verb|latexmk|.
	\item Auto-completion of commands from used standard libraries as well your references in the document.
	\item Good integration of the log, jumping to the line that throws the error or warning.
	\item Seamless handling of multiple source files for a document.
\end{itemize}