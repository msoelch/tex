\section{Version Control}\label{sec:versioncontrol}
It is generally recommended to use version control for your source code.
Following some rules makes using version control with \TeX\ documents that much smoother:

Put each sentence on its own line in the source code, it makes comparing changes much easier.

Split large documents into multiple files.
One file per section, integrated into the main document with \verb|\input| or \verb|\include|, is a good rule of thumb.
That way, changing the order of sections has you change two lines, rather than several hundred insertions and deletions.

Do \emph{not} commit the auxiliary files produced while compiling the document.
Use, \eg, \verb|.gitignore| templates%
\footnote{\url{https://www.toptal.com/developers/gitignore}}
to exclude them.

Do \emph{not} commit the final document.
This will create very large repositories that hold \emph{every} committed revision.
To make the latest pdf easily available, use continuous integration.
With gitlab, it is as easy as adding a file
\begin{Verbatim}[tabsize=4]
build_pdf:
	image: aergus/latex
script:
	- latexmk -pdf -cd relative/path/to/paper.tex
artifacts:
	paths:
		- relative/path/to/paper.pdf
\end{Verbatim}
as \verb|.gitlab-ci.yml|.
The final pdf can then be found at \url{<link-to-repo>/-/jobs/artifacts/master/raw/relative/path/to/paper.pdf?job=build_pdf}.