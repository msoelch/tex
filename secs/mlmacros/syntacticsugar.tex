\section{Syntactic sugar}
\subsection{Math}
Some general, commonly used math expressions are provided by short, readable macros:
\begin{itemize}
	\item \verb|mathbb| wrappers: \verb|\NN| for \NN, \verb|\RR| for \RR, \verb|\PP| for \PP, \verb|\EE| for \EE.
	\item \verb|mathcal| wrappers: \verb|\mcX| for \mcX, \verb|\mcU| for \mcU, \verb|\mcZ| for \mcZ, \verb|\mcD| for \mcD.
	\item \verb|\data| for \data.
	\item \verb|\ind| for \ind.	
	\item \verb|\left\right| wrappers:
		\begin{itemize}
			\item \verb|\abs{}|: $ \abs{\frac a b }$ 
			\item \verb|\set{}|: $ \set{\frac ab }$	
			\item\verb|\interval{}|: $ \interval{\frac ab} $
		\end{itemize}
	\item Divergences: \verb|\divergence[a]{b}{c}| for $\divergence[a]{b}{c}$
	\item Kullback-Leibler divergence: \verb|\kl{p}{q}| for $\kl{p}{q}$. 
	The parentheses adjust to the height of the arguments unless you use \verb|\kl*|.
	\item Gaussian/Normal distributions: \verb|\gauss{0,1}| for $\gauss{0, 1}$. The parentheses adjust to the height of the arguments unless \verb|\gauss*| is used. 
	Can be used for conditionals as well: \verb+\gauss{x | \mu, \sigma}+ gives $\gauss{x | \mu, \sigma}$.
	\item Integrals: \verb|\dint| for nicer integrals.
	
		\verb|\int x \dint x| for $\int x \dint x$ instead of \verb|\int x dx|:$\int x dx$
	\item Equations that are to be proven: \verb|x \shallbe y| for $x \shallbe y$. 
	\item Transpositions: \verb|A\transpose| for $A\transpose$ (with minor spacing adjustments compared to $A^\top$).
	\item Cancelling: use \verb|\cancelto{a}{b}| for $\cancelto{a}{b}$
	\item Superscripts: use \verb|x\super{i}| for $x\super{i}$
	\item Gradients: use \verb|\grad\_{\theta}| for $\grad_{\theta}$ for better spacing than $\nabla_\theta$
\end{itemize}
\subsection{Abbreviations}
One of the most cumbersome typesetting issues is the correct spacing after abbreviations.
The following abbreviations are provided by \verb|mlmacros|:
\begin{center}
	\begin{tabular}{lll}
		\textsc{Abbrev.} & \textsc{Macro} & \textsc{Meaning}\\[.5em]
		\cf & \verb|\cf| & conferatur; compare\\
		\dof & \verb|\dof| & degrees of freedom\\
		\eg, \Eg & \verb|\eg|, \verb|\Eg| & exempli gratia; for instance\\
		\etal & \verb|\etal| & et alii; and others\\
		\etc & \verb|\etc| & et cetera\\
		\ie, \Ie & \verb|\ie|, \verb|\Ie| & id est; that is\\
		\iid & \verb|\iid| & independent and identically distributed\\
		\NB & \verb|\NB| & Nota Bene; note well\\
		\vs & \verb|\vs| & versus\\
		\wrt & \verb|\wrt| & with respect to
	\end{tabular}
\end{center}
All these macros check for a following dot.

To understand the difference, compare the spacing in

\begin{tabular}{l}
	\verb|There are examples, e.\ g.\ this, that and the other.|\\
	There are examples, e. g. this, that and the other.\\
	\verb|There are examples, \eg this, that and the other.|\\
	There are examples, \eg this, that and the other.
\end{tabular}

\subsection{Other}
For easy use in, \eg, tables: \verb|\xmark| and \verb|\cmark| for \xmark and \cmark.