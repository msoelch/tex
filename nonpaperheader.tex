% These packages are very nice features for a note.
% However, they are typically incompatible with a conference style.
% Hence, they are separated into a second header file

% % % % % % % % % %
% FONTS & STYLE
% % % % % % % % % %
\usepackage[osf,sc]{mathpazo} % Palatino as the main font
\linespread{1.05}\selectfont % Palatino needs some extra spacing, here 5% extra
%%\usepackage[scaled=0.85]{beramono}
\usepackage[euler-digits]{eulervm} % nicer math font
\renewcommand{\mathbf}{\mathbold} % necessary for greek letters in this math font to be displayed correctly
\usepackage[parfill]{parskip} % blank lines between paragraphs instead of indent

\usepackage[
%showframe,
paperwidth=210mm,
paperheight=297mm,
left=80pt,
top=80pt,
textwidth=345pt,
marginparsep=20pt,
marginparwidth=129pt,
textheight=692pt,
footskip=50pt
]
{geometry}

% % % % % % % % % %
% BIBLIOGRAPY
% % % % % % % % % %


\usepackage[
	backend=biber, %instead of bibtex
	%backend=bibtex8,
	bibencoding=ascii,%
	language=auto,%
	style=authoryear,%
	%style=authoryear-comp, % Author 1999, 2010
	%bibstyle=authoryear,dashed=false, % dashed: substitute rep. author with ---
	sorting=nyt, % name, year, title
	maxbibnames=4, % default: 3, et al.
	%backref=true,%
	natbib=true % natbib compatibility mode (\citep and \citet still work)
]{biblatex}
% alternative: \usepackage{natbib}

\addbibresource{example.bib}