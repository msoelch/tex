\usepackage[orthodox]{nag}


% % % % % % % % % %
% MATH
% % % % % % % % % %
\usepackage[fleqn]{amsmath}	% fleqn for left alignment of math blocks (as in classicthesis)
\usepackage{amssymb}
\usepackage{amsfonts}
\usepackage{amsthm}
\newtheorem{definition}{Definition}%[chapter]
\newtheorem{theorem}{Theorem}%[chapter]
\newtheorem{corollary}{Corollary}%[chapter]


% % % % % % % % % %
% FLOATS, FIGURAES, AND TABLES
% % % % % % % % % %
\usepackage{graphicx} 		% Required for including images
\graphicspath{{gfx/}}
\usepackage{caption}
\captionsetup{font=small} 	% format=hang,
\usepackage{subfig}
\usepackage{tabularx}
\setlength{\extrarowheight}{3pt} % increase table row height
\usepackage{pbox} 			% for linebreaking in cells
\usepackage{multicol, multirow}
\usepackage{makecell}
\usepackage{tikz}
\usepackage{tikzsymbols}
\usetikzlibrary{fit,arrows,shapes}
\usetikzlibrary{decorations.pathreplacing}
\usetikzlibrary{shadows}
\tikzset{>=latex}
\tikzset{every picture/.style=thick}


% % % % % % % % % %
% MISC
% % % % % % % % % %
\usepackage
[
%disable
]{todonotes}
\usepackage[utf8]{inputenc} % allow utf-8 input
\usepackage[T1]{fontenc}    % use 8-bit T1 fonts
%\usepackage[english]{babel} % hyphenation, special characters, ...
\usepackage{csquotes}		% recommended with babel for quotes
\usepackage[hidelinks,breaklinks]{hyperref}       % hyperlinks
\usepackage{cleveref}		% provides \cref for nice in-document refs
\usepackage{url}            % simple URL typesetting
\usepackage{nicefrac}       % compact symbols for 1/2, etc.
\usepackage{booktabs}
%\usepackage
%[
%activate={true,nocompatibility}, % activate protrusion and expansion
%final,						% enable microtype; use "draft" to disable
%tracking=true,
%kerning=true,
%spacing=true,
%factor=1100,				% add 10% to the protrusion amount (default: 1000)
%stretch=10,					% reduce stretchability (default: 20)
%shrink=10					% reduce shrinkability (default: 20)
%]
%{microtype}
%\microtypecontext{spacing=nonfrench}
\usepackage{cancel}			% easy cancelling of terms: \cancel{expression}
\usepackage[shortcuts]{extdash} % use, e.g., \-/ for hyphen within a word to indicate that the word may be broken here
\usepackage[printonlyused]{acronym}
\renewcommand{\acffont}[1]{\emph{#1}}
\renewcommand{\acfsfont}[1]{\normalfont #1}
\newcommand{\accite}[2]{\acffont{\acl{#1}} (\acsu{#1}; \cite{#2})\xspace}
\newcommand{\acpcite}[2]{\acffont{\aclp{#1}} (\acsp{#1}\acused{#1}; \cite{#2})\xspace}