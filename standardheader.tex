\usepackage[l2tabu, orthodox]{nag}


% % % % % % % % % %
% MATH
% % % % % % % % % %
\usepackage[fleqn]{amsmath}	% fleqn for left alignment of math blocks (as in classicthesis)
\usepackage{amssymb}
\newcommand{\eq}[1]{\begin{align*}#1\end{align*}}
\newcommand\numberthis{\addtocounter{equation}{1}\tag{\theequation}}


% % % % % % % % % %
% FONTS
% % % % % % % % % %
\usepackage[osf,sc]{mathpazo} % Palatino as the main font
\linespread{1.05}\selectfont % Palatino needs some extra spacing, here 5% extra
%\usepackage[scaled=0.85]{beramono}
\usepackage[euler-digits]{eulervm} % nicer math font


% % % % % % % % % %
% FLOATS, FIGURAES, AND TABLES
% % % % % % % % % %
\usepackage{graphicx} 		% Required for including images
\usepackage{caption}
\captionsetup{font=small} 	% format=hang,
\usepackage{subfig}
\usepackage{tabularx}
\setlength{\extrarowheight}{3pt} % increase table row height
\usepackage{pbox} 			% for linebreaking in cells
\usepackage{multicol, multirow}
\usepackage{tikz}
\usetikzlibrary{fit,arrows,shapes}
\usetikzlibrary{decorations.pathreplacing}



% % % % % % % % % %
% BIBLIOGRAPY
% % % % % % % % % %


\usepackage[
	backend=biber, %instead of bibtex
	%backend=bibtex8,
	bibencoding=ascii,%
	language=auto,%
	style=numeric-comp,%
	%style=authoryear-comp, % Author 1999, 2010
	%bibstyle=authoryear,dashed=false, % dashed: substitute rep. author with ---
	sorting=nyt, % name, year, title
	maxbibnames=10, % default: 3, et al.
	%backref=true,%
	natbib=true % natbib compatibility mode (\citep and \citet still work)
]{biblatex}
\usepackage{biblatex}
% alternative: \usepackage{natbib}


% % % % % % % % % %
% MISC
% % % % % % % % % %
\usepackage
[
%disable
]{todonotes}
\usepackage[utf8]{inputenc} % allow utf-8 input
\usepackage[T1]{fontenc}    % use 8-bit T1 fonts
\usepackage[english]{babel} % hyphenation, special characters, ...
\usepackage{csquotes}		% recommended with babel for quotes
\usepackage{hyperref}       % hyperlinks
\usepackage{cleveref}		% provides \cref for nice in-document refs
\usepackage{url}            % simple URL typesetting
\usepackage{nicefrac}       % compact symbols for 1/2, etc.
\usepackage
[
activate={true,nocompatibility}, % activate protrusion and expansion
final,						% enable microtype; use "draft" to disable
tracking=true,
kerning=true,
spacing=true,
factor=1100,				% add 10% to the protrusion amount (default: 1000)
stretch=10,					% reduce stretchability (default: 20)
shrink=10					% reduce shrinkability (default: 20)
]
{microtype}
\usepackage{cancel}			% easy cancelling of terms: \cancel{expression}